%%%%%%% Document settings %%%%%%%%%%
\documentclass{DESSThesis}

%%%%%%%%%%% Additional Packages %%%%%%%%%%%%%%%%%%%%%%%%%%%
% Some suggestions are added, simply uncomment
%Highlight todos within text
% \usepackage[colorinlistoftodos,prependcaption, textsize=tiny]{todonotes} 
% \newcommandx{\note}[2][1=]{\todo[inline, size = \small, backgroundcolor=SeaGreen!25, bordercolor=PineGreen, #1]{#2}}
% \usepackage{algorithm} %Algorithm environment, requires texlive-science
% \usepackage{listings}
% \usepackage{algpseudocode} %Pseudocode commands used in algorithm
% \renewcommand{\algorithmicrequire}{\textbf{Input: }}
% \renewcommand{\algorithmicensure}{\textbf{Output: }}



%%%%%%%%%%% Personalized commands and definitions %%%%%%%%%
\newcommand{\bigO}[1]{\ensuremath{\mathcal{O}\big(#1\big)}}
\newtheorem*{theorem}{Theorem}
\DeclareMathOperator{\supp}{supp} %prevents mathmode to change the font
%Plot own functions using tikz and pgfplot
\pgfplotsset{compat=1.13}
\pgfmathdeclarefunction{gauss}{2}{%
  \pgfmathparse{1/(#2*sqrt(2*pi))*exp(-((x-#1)^2)/(2*#2^2))}%
}

%%%%%%%%%%%%%%%%%%%%%% Hyphenations %%%%%%%%%%%%%%
\hyphenation{trac-ta-ble}

%%%%%%%%%%%%%%%%%%%%%%%%%%%%%%%%%%%%%%%%%%%%%%%%%%

\overfullrule=2cm %Show hboxes incase of misalignment, uncomment before handing in!

\bibliography{references}

\begin{document}

%%%%%%%%%%%%%%%%%%%%% Frontpage %%%%%%%%%%%%%%%%%%%%%%%%%%%%%%% 
\def \TypeofThesis{Master Thesis}
\def \TitleofThesis{Something interesting and meaningful}
\def \AuthorofThesis{Your Name}
\def \FirstSupervisor{Your first Supervisor}
\def \SecondSupervisor{Your second Supervisor}
\def \Advisor{Advisor}
%%%%%%%%%%%%%%%%%%%%%%%%%%%%%%%%%%%%%%%%%
% Academic Title Page
% LaTeX Template
% Version 2.0 (17/7/17)
%
% This template was downloaded from:
% http://www.LaTeXTemplates.com
%
% Original author:
% WikiBooks (LaTeX - Title Creation) with modifications by:
% Vel (vel@latextemplates.com)
% Lorena Reintgen
%
% License:
% CC BY-NC-SA 3.0 (http://creativecommons.org/licenses/by-nc-sa/3.0/)
% 
%
%%%%%%%%%%%%%%%%%%%%%%%%%%%%%%%%%%%%%%%%%

%----------------------------------------------------------------------------------------
% TITLE PAGE
%----------------------------------------------------------------------------------------

\begin{titlepage} % Suppresses displaying the page number on the title page and the subsequent page counts as page 1
\newcommand{\HRule}{\rule{\linewidth}{0.5mm}} % Defines a new command for horizontal lines, change thickness here
\center % Centre everything on the page
%\renewcommand{\baselinestretch}{1.2} %vertical spacing between lines
%------------------------------------------------
%  Headings
%------------------------------------------------
\textsc{\LARGE \TypeofThesis}\\[1.5cm] % Main heading such as the name of your university/college
%------------------------------------------------
%  Title
%------------------------------------------------
\HRule\\[0.4cm]
{\huge\bfseries \TitleofThesis}\\[0.4cm] % Title of your document
\HRule\\[1.5cm]
%------------------------------------------------
%  Author(s)
%------------------------------------------------
    \vfill
    \textit{\large submitted by}\\[0.5cm] % Major heading
\textsc{\Large \AuthorofThesis}\\[0.5cm] % Minor heading
    \vfill

{\large\textit{Submitted to the}}\\
\Large Chair for Data Science in the Economic and Social Sciences \\
    {\large\textit{within the}}\\
    Faculty of Business Administration \\
    at University of Mannheim 
    
    %------------------------------------------------
%  Date
%------------------------------------------------
\vfill\vfill\vfill % Position the date 3/4 down the remaining page
{\large\today} % Date, change the \today to a set date if you want to be precise

\vfill\vfill\vfill
{ \large \center 
\textit{Advisor:}\\
\Advisor
}
\vfill\vfill\vfill
{ \large \center 
\textit{Supervisor:}\\
\FirstSupervisor
}
% \vfill
%     \begin{minipage}{0.48\textwidth}
% \begin{flushleft}
% \large
% \textit{First Supervisor}\\
% \FirstSupervisor
% \end{flushleft}
% \end{minipage}
% ~
% \begin{minipage}{0.48\textwidth}
% \begin{flushright}
% \large
% % \textit{Second Supervisor}\\
% % \SecondSupervisor
% \end{flushright}
% \end{minipage}

%------------------------------------------------
%  Logo
%------------------------------------------------
%\vfill\vfill
%\includegraphics[width=0.2\textwidth]{placeholder.jpg}\\[1cm] % Include a department/university logo - this will require the graphicx package
 
%----------------------------------------------------------------------------------------
\vfill % Push the date up 1/4 of the remaining page
\end{titlepage}

%\vfill
% \cleardoublepage

\tableofcontents
%%%%%%%%%%%%%%%%%%%%% Abstract %%%%%%%%%%%%%%%%%%%%%%%%%%%%% 
\newpage
\thispagestyle{plain}
\begin{abstract}
	Hello, here is some text without a meaning. This text should show what a printed text will look like
at this place. If you read this text, you will get no information. Really? Is there no information?
Is there a difference between this text and some nonsense like “Huardest gefburn”? Kjift – not at
all! A blind text like this gives you information about the selected font, how the letters are written
and an impression of the look. This text should contain all letters of the alphabet and it should be
written in of the original language. There is no need for special content, but the length of words
should match the language.
\end{abstract}
\cleardoublepage
\pagenumbering{arabic} %pagenumbering always resets the page number
%%%%%%%%%%%%%%%%%%%%% Chapter 1 %%%%%%%%%%%%%%%%%%%%%% 
\chapter{Introduction}
\thispagestyle{empty}
Hello, here is some text without a meaning. This text should show what a printed text will look like
at this place. If you read this text, you will get no information. Really? Is there no information?
Is there a difference between this text and some nonsense like “Huardest gefburn”? Kjift – not at
all! A blind text like this gives you information about the selected font, how the letters are written
and an impression of the look. This text should contain all letters of the alphabet and it should be
written in of the original language. There is no need for special content, but the length of words
should match the language.
It also shows that citations work in fluent text \cite{lit:koblergraphisomorphism}
\footnote{or in a footnote \cite{groheparameterized}}
\section{Section}
\begin{figure}[htb!]
	\centering
	\includegraphics[width=6cm]{img/thing.jpg}
	\caption{This graphic visualizes something really interesting.}
	\label{fig:dummy}
  \end{figure}
\paragraph{Paragraph - Formulas}
This is a regular formula $ x^{2 \pi}$ and a formula with a linebreak 
\[ \min_{\mathbf{Q}} \sum_i \left\| \mathbf{A}_i \mathbf{Q} - \mathbf{P}_i(\mathbf{A}_i \mathbf{Q}) \right\|^2_\mathrm{F} \]
Another option:
$$
x = -\frac{p}{2} \pm \sqrt{(\frac{p}{2})^2 -q}
$$

Or also as a numbered equation:
\begin{equation}
	x = -\frac{p}{2} \pm \sqrt{(\frac{p}{2})^2 -q}
\end{equation}

Or in alignment with multiple equations:
\begin{align*}
	\psi(x,y) = [lfp R, v. &(v = y) \vee \\
        &(V_0 v \wedge \exists v' (Evv' \wedge Rv')) \vee\\
        &(V_1 v \wedge \forall v' (Evv' \rightarrow Rv'))](y)
\end{align*}

\[F(X) = \begin{cases}
	A & \mbox{ for } X = \emptyset\\
	X & \mbox{ otherwise}
\end{cases}\]

Something fancy:
\begin{align*}
	\pi  &= \overbrace{11}^{x_0}|\overbrace{0}^{x_1}|\overbrace{101}^{x_2}|\overbrace{100 0}^{x_3}|\overbrace{101}^{x_4}|\overbrace{00 0}^{x_5}|\overbrace{1}^{x_6}|\ldots\\
	\pi' &= \underbrace{11}_{x_0'}|\underbrace{1 010}_{x_1'}|\underbrace{011}_{x_2'}|\underbrace{1 010}_{x_3'}|\underbrace{11}_{x_4'}|\underbrace{1 0}_{x_5'}|\ldots
\end{align*}

and with some text in same row:
\begin{equation*}
    w^T \phi(x) + b =0 \quad \Rightarrow \text{Nonlinear Classifier in} \; R^D
\end{equation*}

% for better overview chapters can be included from separate files
This is an external file and needs no further declarations

\section{Some drawing and tables}
A simple automaton drawn with tikz nodes.\\ 
\begin{tikzpicture}
	[state/.style ={draw, circle, inner sep = 0pt}]
			\node[state] (0) at (0,0) {$0$};
			\node[state] (1) at (3,0) {$1$};
			\path[->] (0) edge[loop above, font = \footnotesize] node {1,2,3,4,5,7} (0)
				edge[bend left, font = \footnotesize] node[above] {6} (1)
				(1) edge[loop above] node[font = \footnotesize,above] {1,2,4,5,6,7} (1)
				edge[bend left] node[font = \footnotesize, below] {3} (0);
    \end{tikzpicture}

A table containing some information:
\begin{center}
	\begin{tabular}{c|c|c}
		\ & $\ell$ & $r$\\
		\hline
		$t$ & $(0, 0)$ & $(0, 0)$\\
		\hline
		$b$ & $(0, 0)$ & $(1, 1)$
	\end{tabular}
\end{center}
And an annotated table:
\[ \begin{blockarray}{ c c c c}
	& denied & granted & \sum \\
	\begin{block}{ c(cc)c }
	  protected & a & b &  n_{p}  \\
	  unprotected & c & d & n_{u} \\
	\end{block}
	\begin{block}{c c c c}
		\sum & m_d & m_g & n\\
	\end{block}
  \end{blockarray}\]%

Plotting can also be done in a tikz environment:\\
\begin{tikzpicture}[scale = 0.6]
	\begin{axis}[every axis plot post/.append style={
		mark=none,domain=1:10,samples=50,smooth},
	  axis x line*=bottom, % no box around the plot, only x and y axis
	  axis y line*=left, % the * suppresses the arrow tips
	  ylabel = {$Pr(\lambda' = x)$},
	  xtick={5},
	  minor xtick={2,4,...,10},
	  xticklabels = {$\lambda$},
	  grid = major, ymajorgrids = false,
	  x tick label style={black},
	  enlarge y limits={abs value=.15, upper}, ]
	  \addplot {gauss(5,0.75)};
	  \legend{Distribution of the sample mean}
	\end{axis}
	\pgftransformshift{\pgfpoint{10cm}{0cm}}
	\begin{axis}[every axis plot post/.append style={
		mark=none,domain=1:10,samples=50,smooth},
		axis x line*=bottom, % no box around the plot, only x and y axis
		axis y line*=left, % the * suppresses the arrow tips
		ylabel = {$Pr(\lambda' = x)$},
		xtick={3.33,5},
		minor xtick={2,4,...,10},
		xticklabels = {$\frac{2}{3}\lambda$,$\lambda$},
		grid = major, ymajorgrids = false,
		x tick label style={black},
		enlarge y limits={abs value=.15, upper}, ]
		\addplot[color = red] {gauss(5*0.666,0.75)};
		\legend{Mixed strategy}
	  \end{axis}
	\end{tikzpicture}
%%%%%%%%%%%%%%%%%%%%% Bibliography %%%%%%%%%%%%%%%%%%%%%% 
\addcontentsline{toc}{chapter}{Bibliography}
\printbibliography
\end{document}  